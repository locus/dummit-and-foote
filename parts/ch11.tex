\chapter{Vector Spaces}

\section{Definitions and Basic Theory}

\answer{1}{
Suppose $(a_1,a_2,\dots,a_n)$ is not the zero vector. Let $\varphi:\mathbb{R}^n\to \mathbb{R}$ be given by $\varphi(x_1,x_2,\dots,x_n)=a_1x_1+a_2x_2+\dots+a_nx_n$. This is a linear mapping because
\begin{align*}
\varphi((x_1,x_2,\dots,x_n)+(y_1,y_2,\dots,y_n)) &= a_1(x_1+y_1)+a_2(x_2+y_2)+\dots+a_n(x_n+y_n) \\
&= (a_1x_1+a_2x_2+\dots+a_nx_n)+(a_1y_1+a_2y_2+\dots+a_ny_n) \\
&= \varphi(x_1,x_2,\dots,x_n)+\varphi(y_1,y_2,\dots,y_n)
\end{align*}
and
\begin{align*}
\varphi(\alpha(x_1, x_2,\dots, x_n)) &= a_1(\alpha x_1)+a_2(\alpha x_2)+\dots+a_n(\alpha x_n) \\
&= \alpha(a_1x_1+a_2x_2+\dots+a_nx_n) \\
&= \alpha\varphi(x_1,x_2,\dots,x_n).
\end{align*}
It is clear that $\varphi$ is surjective: pick any $y\in\mathbb{R}$. Without loss of generality, $a_1\neq 0$, hence $y=\varphi((y/a_1),0,\dots,0)$. Thus $\dim\im\varphi=1$. By Corollary 8, we get that $\dim\ker\varphi = n-1$. Since the kernel of $\varphi$ is the set of elements $(x_1,x_2,\dots,x_n)$ of $\mathbb{R}^n$ with $a_1x_1+a_2x_2+\dots+a_nx_n=0$ and $\ker\varphi$ is a subspace of $\mathbb{R}^n$, this answers the first part of the question.

To find a base, let
\begin{equation*}
B = \left\{b_2=\left(\frac{-a_2}{a_1},1,0,\dots,0\right), b_3=\left(\frac{-a_3}{a_1},0,1,\dots,0\right), \dots, b_{n}=\left(\frac{a_n}{a_1},0,0,\dots,1\right)\right\}
\end{equation*}
(recall that $a_1\neq 0$). Pick any vector $x=(x_1,x_2,\dots,x_n)\in\ker\varphi$. Then $x_1 = (-1/a_1)(a_2x_2+\dots+a_nx_n)$, hence $x=x_2b_2+x_3b_3+\dots+x_nb_n$. Thus $B$ spans $\ker\varphi$. Moreover it is easy to see that $B$ is a linearly independent set. Therefore it is a basis (of $n-1$ elements) for the vector subspace $\ker\varphi$.
}

\answer{3}{
Let $b_1=(1,0,0,0)$, $b_2=(1,-1,0,0)$, $b_3=(1,-1,1,0)$ and $b_4=(1,-1,1,-1)$. Then, $(0,1,0,0)=b_2-b_1$, $(0,0,1,0)=b_3-b_2$ and $(0,0,0,1)=b_3-b_4$. Therefore
\begin{align*}
\varphi((a,b,c,d)) &= \varphi(ab_1+b(b_2-b_1)+c(b_3-b_2)+d(b_3-b_4)) \\
&= (a-b)\varphi(b_1)+(b-c)\varphi(b_2)+(c+d)\varphi(b_3)+d\varphi(b_4) \\
&= a-b+c+d.
\end{align*}
This is a concrete realization of an "extension by linearity" of $\varphi$.
}

\section{The Matrix of a Linear Transformation}

\answer{11}{
\begin{enumerate}[label=(\alph*)]
\item Take $y\in\text{im}\,\varphi \cap \ker \varphi$. Then there is some $x\in V$ such that $\varphi(x)=y$ and moreover $\varphi(y)=0$. Since $\varphi(y)=\varphi^2(x)$ and $\varphi=\varphi^2$, we must have $\varphi(y)=\varphi(x)=0$. Hence $y=0$ by the first equality. Because $y$ was arbitrary, we must have $\text{im}\,\varphi\cap\ker\varphi=0$.

\item Take any $x\in V$. The intuition here is that since $\varphi$ is idempotent, the natural projection $\pi:V\to V/\ker\varphi$ will map $x$ and $\varphi(x)$ to the same element, because $x$ and $\varphi(x)$ will basically "collapse" to the same element of $V$. More precisely, we have that $\varphi(x)=\varphi^2(x)$ and thus $\varphi(x-\varphi(x))=0$, or in other words $x-\varphi(x)\in\ker\varphi$. Now $x=\varphi(x)+[x-\varphi(x)]\in\text{im}\,\varphi+\ker\varphi$. Because $x$ was arbitrary, we get that $V=\text{im}\,\varphi+\ker\varphi$. Since $\text{im}\,\varphi$ and $\ker\varphi$ are both submodules of the $F$-module $V$, we get by (a) and by Proposition 5 (on p.329 of D\&F) that $V=\text{im}\,\varphi\oplus\ker\varphi$.

\item Let $\mathcal{B}=\{u_1,u_2,\dots,u_k\}$ be a basis for $\text{im}\,\varphi$ and let $\mathcal{C}=\{v_1,v_2,\dots,v_l\}$ be a basis for $\ker\varphi$. By (b), we know that any $x\in V$ can be written uniquely as $x=\alpha+\beta$ for $\alpha\in\text{im}\,\varphi$ and $\beta\in\ker\varphi$. Therefore any $x\in V$ can also be written uniquely as $x=(a_1u_1+a_2u_2+\dots+a_ku_k)+(b_1v_1+b_2v_2+\dots+b_lv_l)$ for $a_i,b_i\in F$. This means that the set $\{u_1,u_2,\dots,u_k,v_1,v_2,\dots,v_l\}=\mathcal{D}$ spans $V$; moreover the uniqueness signifies that $\mathcal{D}$ is actually a set of independent vectors. Hence $\mathcal{D}$ is a basis for $V$.

Now let $u_i$ be one of the basis element of $\text{im}\,\varphi$ in $\mathcal{B}$. Then there exists some $w\in V$ such that $\varphi(w)=u_i$. Using idempotence, $\varphi(w)=\varphi(u_i)=u_i$. Therefore, in the matrix representation of $\varphi$, in the column standing for the basis element $u_i$, there is a single 1 on the $i$th row and zeros everywhere else, for $i$ ranging from $1$ to $\dim(\text{im}\,\varphi)$.

If $v_i$ is one of the basis element of $\ker\varphi$ in $\mathcal{C}$, then obviously $\varphi(v_i)=0$ and as such the column standing for $v_i$ in the matrix representation of $\varphi$ is only zeroes (with $i$ ranging from $\dim(\text{im}\,\varphi)+1$ to $\dim V$).

This shows that the matrix representation of $\varphi$ is a diagonal matrix with only ones and zeroes in it.
\end{enumerate}
}

\answer{35}{
\begin{enumerate}[label=(\alph*)]
\item Take any matrix $M=\bigl(\begin{smallmatrix} a&b \\ c&d\end{smallmatrix}\bigr)$ in $V$. Then obviously $M = a\bigl(\begin{smallmatrix}1&0\\0&0\end{smallmatrix}\bigr)+b\bigl(\begin{smallmatrix}0&1\\0&0\end{smallmatrix}\bigr)+c\bigl(\begin{smallmatrix}0&0\\1&0\end{smallmatrix}\bigr)+d\bigl(\begin{smallmatrix}0&0\\0&1\end{smallmatrix}\bigr)$, hence these four matrices form a spanning set of $V$. It is also obvious that any smaller subset of these four matrices would not span $V$. Therefore they form a basis for $V$. As a consequence we have $\dim V=4$.

\item Let $\gamma\in\mathbb{R}$ and $A=\bigl(\begin{smallmatrix} a&b \\ c&d\end{smallmatrix}\bigr)\in V$. Then $\varphi(\gamma A) = \gamma a + \gamma d = \gamma(a+d) = \gamma\varphi(A)$. Now let $B=\bigl(\begin{smallmatrix} w&x \\ y&z\end{smallmatrix}\bigr)\in V$. We have $\varphi(A+B)=(a+w)+(d+z)=(a+d)+(w+z)=\varphi(A)+\varphi(B)$. Hence $\varphi$ is a linear transformation between $V$ and $\mathbb{R}$.

Obviously $\mathbb{R}$ is a one-dimensional vector space with $\{1\}$ as a basis. Because $\varphi\bigl(\begin{smallmatrix}1&0\\0&0\end{smallmatrix}\bigr)=\varphi\bigl(\begin{smallmatrix}0&0\\0&1\end{smallmatrix}\bigr)=1$ and $\varphi\bigl(\begin{smallmatrix}0&1\\0&0\end{smallmatrix}\bigr)=\varphi\bigl(\begin{smallmatrix}0&0\\1&0\end{smallmatrix}\bigr)=0$, we get that the matrix representation of $\varphi$ for these bases is given by $M=\bigl(\begin{matrix}1&0&0&1\end{matrix}\bigr)$. Since $\varphi$ is visibly surjective, we have $\dim\varphi(V)=\dim\mathbb{R}=1$.

We know that $\dim V=\dim\varphi(V)+\dim(\ker\varphi)$, and thus $\dim(\ker\varphi)=4-1=3$. Consider the set of matrices $\mathcal{B}=\{\bigl(\begin{smallmatrix}-1&0\\0&1\end{smallmatrix}\bigr),\bigl(\begin{smallmatrix}0&1\\0&0\end{smallmatrix}\bigr),\bigl(\begin{smallmatrix}0&0\\1&0\end{smallmatrix}\bigr)\}$. These three matrices are clearly linearly independent. Because for any real numbers $a,b,c$ we have $\varphi(a\bigl(\begin{smallmatrix}-1&0\\0&1\end{smallmatrix}\bigr)+b\bigl(\begin{smallmatrix}0&1\\0&0\end{smallmatrix}\bigr)+c\bigl(\begin{smallmatrix}0&0\\1&0\end{smallmatrix}\bigr))=a\varphi(\bigl(\begin{smallmatrix}-1&0\\0&1\end{smallmatrix}\bigr))+b\varphi(\bigl(\begin{smallmatrix}0&1\\0&0\end{smallmatrix}\bigr))+c\varphi(\bigl(\begin{smallmatrix}0&0\\1&0\end{smallmatrix}\bigr))=0$, this means that $\text{Span}(\mathcal{B})\subseteq\ker\varphi$. But $\ker\varphi$ has dimension three, as computed earlier. This means that $\mathcal{B}$ is actually a basis for $\ker\varphi$.
\end{enumerate}
}

\section{Dual Vector Spaces}

\answer{4}{
Let $v^*:V\to K$ a linear functionnal of $V^*$ be given by
\begin{equation*}
v^*(v) = \sum_{a^*\in A^*}a^*(v).
\end{equation*}
This sum always has a value (in $K$) because only finitely many values in the sum are 1 (the rest are zeroes). It is easy to see that $v^*$ cannot be written as a finite linear combination of elements of $A^*$. Thus $v^*\notin \Span(A^*)$. This means that $\dim V<\dim V^*$.
}
